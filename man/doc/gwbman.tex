\documentclass[12pt]{article}

\usepackage{ucs}
\usepackage[utf8x]{inputenc} 		% Включаем поддержку UTF8
\usepackage[russian]{babel}  		% Включаем пакет для поддержки русского языка

\usepackage[
	left=2cm, 			% Поле левое : 200 мм
	right=2cm, 			% Поле правое : 200 мм
	top=2cm,			% Поле верхнее: 200 мм
	bottom=2cm,			% Поле нижнее : 200 мм
	bindingoffset=0cm]{geometry}

\usepackage[pdftex]{graphicx, color}
\usepackage{color}
\usepackage{tikz}
\usepackage{url}			% использование URL в библиографии
\usepackage{listings}			% использование листингов кода
\usepackage[nooneline]{caption} 
\captionsetup[table]{justification=raggedleft} 
\captionsetup[figure]{justification=centering,labelsep=endash}
\usepackage{array}

\usepackage{caption}
\usepackage{graphicx}
\usepackage{subcaption}
\usepackage{cases}

\renewcommand{\baselinestretch}{1.5}

% вставка листингов с кодом
\lstset{inputencoding=utf8x,
		extendedchars=false,
		keepspaces=true,
		language=c}

\renewcommand{\lstlistingname}{Листинг}

%\usepackage{amsmath}
%\usepackage{mathtools}
\setcounter{tocdepth}{4} 	% chapter, section, subsection, subsubsection и paragraph
\setcounter{secnumdepth}{4}

\parindent=1,25cm				% красная строка = 1 см
\usepackage{enumitem}
\setlist[enumerate,1]{leftmargin=2.25cm}
\setlist[itemize]{leftmargin=2.25cm}
\graphicspath{{pics/}}
\DeclareGraphicsExtensions{{.jpg}}


\begin{document}
	\tableofcontents %Содержание
	
	\section{Введение}

	\section{Обзор используемых технологий}	
		\subsection{Среда разработки GWBasic}
			% Этот раздел описать преимущественно по Ingleese
			\subsubsection{Режимы работы (Выполнение команд)}
			% Direct, Indirect
			\subsubsection{Cинтаксические конструкции}
			% Commands, Statements, Variables, Expressions, ...
			\subsubsection{Работа с графикой}
			% SCREEN 12
		\subsection{Flex -- инструмент разработки лексического анализатора}
		\subsection{Bison -- инструмент разработки синтаксического анализатора}
		\subsection{Glut -- инструментарий для работы с OpenGL}
	\section{Разработка среды GWBasic}
		\subsection{Анализаторы языка GWBasic}
			% какие анализаторы будут разработаны?
			\subsubsection{Лексический анализатор}
				% элементы лексического анализатора (название команд, операторов, ф-ций, ...)
			\subsubsection{Синтаксический анализатор}
				% Основываясь на 1.1 (Ingleese GWBasic), описать что есть Direct/Indirect Mode
				% Конструкции языка (Операторы, выражения, команды). а затем, из чего они состоят
		\subsection{Архитектура системы исполнения}
			% диаграмма, поэтапная, расширяющаяся
			% обработчики
		\subsection{Среда разработки в графическом режиме}
			% картинка с модулями обработки прерываний с клавиатуры
			% выводом данных
			% представлением терминала
			% написать про структуру ide, и что она в себя включает Env системы исполнения (Runtime)
	\section{Реализация среды GWBasic}
		% Реализация среды на C (К вопросу о языке разработки), 
		\subsection{Лексический и синтаксический анализатор}
			% Использование инструментов Flex и Bison
			% Ссылки на соответствующие пункты разработки + в приложении разместить грамматику
		\subsection{Система исполнения}
			% Описание этих компонентов со ссылкой нареализацию на C.
			% Про обработчики
			% Вывод результата (DisplayDebug, DisplayResult, DisplayCoreValue)
		\subsection{Пользовательский интерфейс}
			% Использование Glut, описать что все элементы 1.3 c точки зрения реализации на Glut
	\section{Тестирование}
		\subsection{Входная программа}

	\section{Заключение}
	
\end{document}
